\documentclass[main.tex]{subfiles}

\section{Question 3}

Write up the order conditions for an embedded Runge-Kutta method with
3 stages. The solution you advance must have order 3 and the embedded
method used for error estimation must have order 2.

\begin{table}[h]
    \centering
    \begin{tabular}{l | c c c }
        0 & 0 & 0 & 0 \\
        $c_2$ & $a_{21}$ & 0 & 0 \\
        $c_3$ & $a_{31}$ & $a_{32}$ & 0 \\
        \hline
        $x$ & $b_1$ & $b_2$ & $b_3$ \\
        $\widehat{x}$ & $\widehat{b}_1$ & $\widehat{b}_2$ & $\widehat{b}_3$ \\
        \hline
        $e$ & $d_1$ & $d_2$ & $d_3$
    \end{tabular}
    \caption{Butcher's tableau schema for Runge-Kutta method with 3 stages.}
\end{table}

\begin{subequations}\label{eq:order_conds}
Order conditions derived from Butcher's table:
\begin{align}
    b^T e &= 1 & &b_1 + b_2 + b_3 = 1 \\
    b^T C e &= \frac{1}{2} & &\underbrace{b_1 c_1}_{0} + b_2 c_2 + b_3 c_3 = \frac{1}{2} \\
    b^T C^2 e &= \frac{1}{3} & &\underbrace{b_1 c_1^2}_{0} + b_2 c_2^2 + b_3 c_3^2 = \frac{1}{3} \\
    b^T A C e &= \frac{1}{6} & &\underbrace{b_2 a_{21} c_1}_{0} + \underbrace{b_3 a_{31} c_1}_{0} + b_3 a_{32} c_2 = \frac{1}{6}
\end{align}
values of $c_2$ and $c_3$ will be set to $\frac{1}{4}$ and $1$ respecively. This leaves us with 6 unknown variables (3 $a$s and 3 $b$s) and only 4 equations so we will add the so called consistency equations.

\begin{align}
    c_2 &= a_{21} \\
    c_3 &= a_{31} + a_{32}
\end{align}
\end{subequations}
Using Matlab to solve the system we get the following results: $b_1 = -\frac{1}{6}$, $b_2 = \frac{8}{9}$, $b_3 = \frac{5}{18}$, $a_{21} = \frac{1}{4}$, $a_{31} = -\frac{7}{5}$, $a_{32} = \frac{12}{5}$.

Next we will solve the system where $c_2$ and $c_3$ are known thus giving 2 equations with 3 unknowns. In order to find a solution $\widehat{b}_2$ is set to be $\frac{1}{2}$.
\begin{subequations}
\begin{align}
    \widehat{b}_1 + \widehat{b}_2 + \widehat{b}_3 &= 1 \\
    \widehat{b}_2 c_2 + \widehat{b}_3 c_3 &= \frac{1}{2}
\end{align}
\end{subequations}
The above system yields $\widehat{b}_1 = \frac{1}{8}$ and $\widehat{b}_3 = \frac{3}{8}$. Going back to the Butcher's tableau we know that last row ($d_1, d_2, d_3$) is just the difference of the previous two rows by definition.

\begin{table}[h]
    \centering
    \begin{tabular}{l | c c c }
        0 & 0 & 0 & 0 \\
        1/4 & 1/4 & 0 & 0 \\
        1 & -7/5 & 12/5 & 0 \\
        \hline
        $x$ & -1/6 & 8/9 & 5/18 \\
        $\widehat{x}$ & 1/8 & 1/2 & 3/8
    \end{tabular}
    \caption{Butcher's tableau with error estimators for our method.}
\end{table}

% TODO:
% 3. corresponding equations defining our method
% 4. test our solver on test eq with lambda=-1
% 5. Verify the order of your method by plotting the local error as function of step size (for the test equation).
% 6. Compute R(hλ) for your method and make a stability plot of your method.
% 7. Test your solver on the van der Pol equation with µ = 3. Compare the solution you get with the solution you get when using ode15s.

